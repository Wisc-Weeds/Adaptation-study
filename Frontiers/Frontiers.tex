%%%%%%%%%%%%%%%%%%%%%%%%%%%%%%%%%%%%%%%%%%%%%%%%%%%%%%%%%%%%%%%%%%%%%%%%%%%%%%%%%%%%%%%%%%%%%%%%%%%%%%%%%%%%%%%%%%%%%%%%%%%%%%%%%%%%%%%%%%%%%%%%%%%%%%%%%%%
% This is just an example/guide for you to refer to when submitting manuscripts to Frontiers, it is not mandatory to use Frontiers .cls files nor frontiers.tex  %
% This will only generate the Manuscript, the final article will be typeset by Frontiers after acceptance.
%                                              %
%                                                                                                                                                         %
% When submitting your files, remember to upload this *tex file, the pdf generated with it, the *bib file (if bibliography is not within the *tex) and all the figures.
%%%%%%%%%%%%%%%%%%%%%%%%%%%%%%%%%%%%%%%%%%%%%%%%%%%%%%%%%%%%%%%%%%%%%%%%%%%%%%%%%%%%%%%%%%%%%%%%%%%%%%%%%%%%%%%%%%%%%%%%%%%%%%%%%%%%%%%%%%%%%%%%%%%%%%%%%%%

%%% Version 3.4 Generated 2018/06/15 %%%
%%% You will need to have the following packages installed: datetime, fmtcount, etoolbox, fcprefix, which are normally inlcuded in WinEdt. %%%
%%% In http://www.ctan.org/ you can find the packages and how to install them, if necessary. %%%

\documentclass[utf8]{frontiersSCNS}

%\setcitestyle{square} % for Physics and Applied Mathematics and Statistics articles
\usepackage{url,hyperref,lineno,microtype,subcaption}
\usepackage[onehalfspacing]{setspace}

\linenumbers


% BELOW TAKEN FROM rticles plos template
%
% amsmath package, useful for mathematical formulas
\usepackage{amsmath}
% amssymb package, useful for mathematical symbols
\usepackage{amssymb}

% hyperref package, useful for hyperlinks
\usepackage{hyperref}

% graphicx package, useful for including eps and pdf graphics
% include graphics with the command \includegraphics
\usepackage{graphicx}

% Sweave(-like)
\usepackage{fancyvrb}
\DefineVerbatimEnvironment{Sinput}{Verbatim}{fontshape=sl}
\DefineVerbatimEnvironment{Soutput}{Verbatim}{}
\DefineVerbatimEnvironment{Scode}{Verbatim}{fontshape=sl}
\newenvironment{Schunk}{}{}
\DefineVerbatimEnvironment{Code}{Verbatim}{}
\DefineVerbatimEnvironment{CodeInput}{Verbatim}{fontshape=sl}
\DefineVerbatimEnvironment{CodeOutput}{Verbatim}{}
\newenvironment{CodeChunk}{}{}

% cite package, to clean up citations in the main text. Do not remove.
\usepackage{cite}

\usepackage{color}

\providecommand{\tightlist}{%
  \setlength{\itemsep}{0pt}\setlength{\parskip}{0pt}}

% Below is from frontiers
%
\bibliographystyle{frontiersinSCNS}
% Use doublespacing - comment out for single spacing
%\usepackage{setspace}
%\doublespacing


% Leave a blank line between paragraphs instead of using \\


\def\keyFont{\fontsize{8}{11}\helveticabold }


%% ** EDIT HERE **
%% PLEASE INCLUDE ALL MACROS BELOW

%% END MACROS SECTION

% Pandoc citation processing
\newlength{\csllabelwidth}
\setlength{\csllabelwidth}{3em}
\newlength{\cslhangindent}
\setlength{\cslhangindent}{1.5em}
% for Pandoc 2.8 to 2.10.1
\newenvironment{cslreferences}%
  {}%
  {\par}
% For Pandoc 2.11+
\newenvironment{CSLReferences}[2] % #1 hanging-ident, #2 entry spacing
 {% don't indent paragraphs
  \setlength{\parindent}{0pt}
  % turn on hanging indent if param 1 is 1
  \ifodd #1 \everypar{\setlength{\hangindent}{\cslhangindent}}\ignorespaces\fi
  % set entry spacing
  \ifnum #2 > 0
  \setlength{\parskip}{#2\baselineskip}
  \fi
 }%
 {}
\usepackage{calc} % for calculating minipage widths
\newcommand{\CSLBlock}[1]{#1\hfill\break}
\newcommand{\CSLLeftMargin}[1]{\parbox[t]{\csllabelwidth}{#1}}
\newcommand{\CSLRightInline}[1]{\parbox[t]{\linewidth - \csllabelwidth}{#1}\break}
\newcommand{\CSLIndent}[1]{\hspace{\cslhangindent}#1}


\def\Authors{
  Maxwel C Oliveira\,\textsuperscript{1},
  Amit J Jhala\,\textsuperscript{2},
  Mark Bernards\,\textsuperscript{3},
  Chris Proctor\,\textsuperscript{2},
  Strahinja Stepanovic\,\textsuperscript{2},
  Rodrigo Werle\,\textsuperscript{1*}}

\def\Address{

  \textsuperscript{1} Department of Agronomy, University of
Wisconsin-Madison,  Madison,  Wisconsin,  United States
  
  \textsuperscript{2} Department of Agronomy and
Horticulture, University of
Nebraska-Lincoln,  Lincoln,  Nebraska,  United States
  
  \textsuperscript{3} Department of Agronomy, Western Illinois
University,  Macomb,  Illnois,  United States
  }

  
  \def\firstAuthorLast{Oliveira {et~al.}}
  
  
  
  
  
  
  
  
  \def\corrAuthor{Rodrigo Werle}\def\corrAddress{Department of Agronomy,
University of Wisconsin-Madison, United States\\1575 Linden
Dr\\Madison, Wisconsin, 53705 United
States}\def\corrEmail{\href{mailto:rwerle@uwisc.edu}{\nolinkurl{rwerle@uwisc.edu}}}
  


\begin{document}
\onecolumn
\firstpage{1}

\title[Short Title]{Adaptation of Palmer amaranth to cropping systems}
\author[\firstAuthorLast]{\Authors}
\address{} %This field will be automatically populated
\correspondance{} %This field will be automatically populated

\extraAuth{}% If there are more than 1 corresponding author, comment this line and uncomment the next one.
%\extraAuth{corresponding Author2 \\ Laboratory X2, Institute X2, Department X2, Organization X2, Street X2, City X2 , State XX2 (only USA, Canada and Australia), Zip Code2, X2 Country X2, email2@uni2.edu}


\maketitle

\begin{abstract}

Abstract length and content varies depending on article type. Refer to 
\url{http://www.frontiersin.org/about/AuthorGuidelines} for abstract requirement
and length according to article type.

%All article types: you may provide up to 8 keywords; at least 5 are mandatory.
\tiny
 \keyFont{ \section{Keywords:} Text Text Text Text Text Text Evolution Weed } 

\end{abstract}

\hypertarget{introduction}{%
\section*{Introduction}\label{introduction}}
\addcontentsline{toc}{section}{Introduction}

Palmer amaranth (\emph{Amaranthus palmeri} S. Watson) is currently
considered one of the most economically damaged weed species to cropping
systems in the United States (Ward et al., 2013). The species has showed
a remarkable capacity to evolve resistance to herbicides. Palmer
amaranth has evolved resistance to eight herbicide sites of action,
increasing the weed management complexity (Lindsay et al., 2017).
Uncontrolled Palmer amaranth in competition for water, light and
nutrients can drastically reduce crop yields (Berger et al., 2015).
Palmer amaranth is documented with potential to reduce 91\%, 68\%, and
54\% of corn (Massinga et al., 2001), soybean (Klingaman and Oliver,
1994), and cotton (Morgan et al., 2001) yields.

Palmer amaranth is a fast growing summer annual forb indigenous to
Sonoran Desert (Sauer, 1957). The species would eventually emerge as a
threat to US agriculture in the 1990s. Palmer amaranth weediness is
likely a result of human-assisted selection in combination with species
biology. Farm mechanization, conservation agriculture (e.g., no-till),
and reliance on herbicides for weed management are the main human
mediated selection of Palmer amaranth to cropping systems. On the other
hand, Palmer amaranth is a prolific seed producer with a C4
photosynthetic apparatus (Ward et al., 2013). With a dioecy nature,
Palmer amaranth male and female plants are obligate outcrosser species,
increasing the chances of exchanging herbicide resistant alleles among
plants (Oliveira et al., 2018). Also, Palmer amaranth small seed size (1
mm) tend to thrive in no-tillage systems (Price et al., 2011), and
spread across locations through farm equipment (Sauer, 1972), manure (Yu
et al., 2021), animals (Farmer et al., 2017), and plant propagules
(Hartzler and Anderson, 2016). Therefore, Palmer amaranth dispersal
capacity make the species one of the most successful cases of weed
adaption to cropping systems.

Light and temperature are likely the main environment requirements for
Palmer amaranth successful adaptation. Palmer amaranth is reported with
an extended germination period (Jha et al., 2010). Germination of Palmer
amaranth is triggered by 18 C soil temperature (Keeley et al., 1987),
and optimal germination and biomass production occur at 35/30 C day and
night temperatures (Guo and Al-Khatib, 2003). Water has not shown to
limit Palmer amaranth fitness. Under continuous water stress, Palmer
amaranth survived and produced at least 14000 seeds plant-1 (Chahal et
al., 2018). Also, seeds from Palmer amaranth growing under water stress
conditions were heavier, less dormant, and prompt for germination
(Matzrafi et al., 2021). The continuous global temperature warming can
impact agriculture and promote niches for Palmer amaranth
invasion/adaptation into new environments. Currently, it is estimated
that the greatest climatic risk of Palmer amaranth establishment is
agronomic crops in Australia and Sub-Sahara Africa (Kistner and
Hatfield, 2018). Temperature is a key factor limiting Palmer amaranth
expansion to cooler geographies (Briscoe Runquist et al., 2019);
however, under future climate change Palmer amaranth is likely to expand
northward into Canada and Northern Europe (Kistner and Hatfield, 2018).

Palmer amaranth is already found in agronomic crops of South America
(Larran et al., 2017; Küpper et al., 2017) and Southern Europe (Milani
et al., 2021). In the US, Palmer amaranth is commonly found at crop
(Garetson et al., 2019) and non-crop land (Bagavathiannan and
Norsworthy, 2016) in the warm southern United States but its range is
expanding to cool temperatures northward. For example, herbicide
resistant Palmer amaranth is widespread in Nebraska (Oliveira et al.,
2021), Michigan (Kohrt et al., 2017), and Connecticut (Aulakh et al.,
2021). Successful cases of Palmer amaranth invasion and near to
eradication is well documented in Minnesota (Yu et al., 2021) and Iowa.
No Palmer amaranth actively growing was found in Canada; however, Palmer
amaranth seeds was detected in sweet potato slips (Page et al., 2021).
Nonetheless, it seems fated to manage Palmer amaranth in agronomic crops
throughout multiple environments in the near future. Therefore,
strategies on Palmer amaranth management should encompass the
agroecosystem level but not attempts to eradicate the weed. Most tactics
to manage Palmer amaranth are based technology fixes (Scott, 2011),
which are short-term (e.g., herbicide and/or tillage) rather than
long-term Palmer amaranth management.

The continuous Palmer amaranth dispersal and potential establishment
into northern United States warrant investigations on species morphology
growing in such environments. Understanding Palmer amaranth biology and
growing strategies under different agroecossystems can enhance our
knowledge on species adaptation. Also, it can aid on designing proactive
and ecological tactics to limit the species range expansion, reduce its
negative impact, and design resilient and sustainable farming systems
(MacLaren et al., 2020). Therefore, the objective of this study was to
investigate the flowering pattern, biomass production, and height of
Palmer amaranth growing under in corn, soybean and fallow at two timings
across five locations in the mid/upper United States Midwest.

\hypertarget{material-and-methods}{%
\section*{Material and Methods}\label{material-and-methods}}
\addcontentsline{toc}{section}{Material and Methods}

\hypertarget{plant-material-and-growing-conditions}{%
\subsection*{Plant material and growing
conditions}\label{plant-material-and-growing-conditions}}
\addcontentsline{toc}{subsection}{Plant material and growing conditions}

The study was performed with a \emph{A. palmeri} accession (Per1) from
Perkins County, Nebraska. Per1 accession collection is documented in
(Oliveira et al., 2021), with no reported herbicide resistance. Three
weeks prior to the field experiment, seeds were planted in plastic trays
containing potting-mix. Emerged seedlings (1 cm) were transplanted into
200 cm-3 plastic pots (a plant pot-1). Palmer amaranth seedlings were
supplied with adequate water and kept under greenhouse conditions at
Arlington, Clay Center, Lincoln, and Macomb; and kept outdoors in Grant.
Palmer amaranth seedlings were kept under greenhouse/outdoors until the
onset of the experiment (7 to 10 cm height).

\hypertarget{field-study}{%
\subsection*{Field study}\label{field-study}}
\addcontentsline{toc}{subsection}{Field study}

The experiment was conducted in 2018 and 2019 under field conditions at
five locations: Arlington (Washington County, Wisconsin), Clay Center
(Clay County, Nebraska), Grant (Perkins County, Nebraska), Lincoln
(Lancaster County, Nebraska), and Macomb (McDonough County, Illinois).

The experimental unit were adjacent 9.1 m wide (12 rows at 72.2 cm row
spacing) by 10.7 m long. Each experimental unit was planted with corn or
soybean, or left fallow. Palmer amaranth seedlings were transplanted to
the field experiment by making a whole in the soil (6 cm deep and 8 cm
wide); and gently transferring in the ground (potting mix + two
seedlings). After a week, if both plants were alive, one was eliminated.
There were two transplant timing: early (June 1\textsuperscript{st}) and
late (July 1\textsuperscript{st}). There were 24 Palmer amaranth plants
in each crop/fallow and timing, with a total of 144 plants. The study
was repeated twice.

After transplanting, Palmer amaranth flowering was monitored until the
end of the study. When a plant started flowering, the day was recorded,
plant sex was identified as male or female, and plant height was
measured from soil surface to the plant top. Then, aboveground plant
biomass was harvest near soil surface and oven dried at 65 C until
reaching constant weight before the weight of biomass (g
plant\textsuperscript{-1}) was recorded.

\hypertarget{statistical-analyses}{%
\subsection*{Statistical analyses}\label{statistical-analyses}}
\addcontentsline{toc}{subsection}{Statistical analyses}

The statistical analyses were performed using R statistical software
version 4.0.1.

The cumulative Palmer amaranth flowering estimation was determined using
a asymmetrical three parameter log logistic Weibull model of the drc
package (Ritz et al., 2015).

\[Y(x) = 0 + (d-0) exp (-exp(b(log(x)-e)))\] In this model, \emph{Y} is
the Palmer amaranth cumulative flowering, \emph{d} is the upper limit
(set to 100), and \emph{e} is the XXX, and \emph{x} day of year (doy).

The doy for 10, 50, and 90\% Palmer amaranth cumulative flowering were
determined using the \emph{ED} function of drc package. Also, the 10,
50, and 90\% Palmer amaranth cumulative flowering were compared among
crop/fallow and timings using the \emph{EDcomp} function of drc package.
The EDcomp function compares the ratio of cumulative flowering using
t-statistics, where P-value \textless{} 0.05 indicates that we fail to
reject the null hypothesis.

\hypertarget{results}{%
\section*{Results}\label{results}}
\addcontentsline{toc}{section}{Results}

\hypertarget{subsection-1}{%
\subsection*{Subsection 1}\label{subsection-1}}
\addcontentsline{toc}{subsection}{Subsection 1}

You can use \texttt{R} chunks directly to plot graphs.

\hypertarget{subsection-2}{%
\subsection*{Subsection 2}\label{subsection-2}}
\addcontentsline{toc}{subsection}{Subsection 2}

Frontiers requires figures to be submitted individually, in the same
order as they are referred to in the manuscript. Figures will then be
automatically embedded at the bottom of the submitted manuscript. Kindly
ensure that each table and figure is mentioned in the text and in
numerical order. Permission must be obtained for use of copyrighted
material from other sources (including the web). Please note that it is
compulsory to follow figure instructions. Figures which are not
according to the guidelines will cause substantial delay during the
production process.

\hypertarget{discussion}{%
\section{Discussion}\label{discussion}}

\hypertarget{disclosureconflict-of-interest-statement}{%
\section*{Disclosure/Conflict-of-Interest
Statement}\label{disclosureconflict-of-interest-statement}}
\addcontentsline{toc}{section}{Disclosure/Conflict-of-Interest
Statement}

The authors declare that the research was conducted in the absence of
any commercial or financial relationships that could be construed as a
potential conflict of interest.

\hypertarget{author-contributions}{%
\section*{Author Contributions}\label{author-contributions}}
\addcontentsline{toc}{section}{Author Contributions}

MCO design, wrote,

The statement about the authors and contributors can be up to several
sentences long, describing the tasks of individual authors referred to
by their initials and should be included at the end of the manuscript
before the References section.

\hypertarget{acknowledgments}{%
\section*{Acknowledgments}\label{acknowledgments}}
\addcontentsline{toc}{section}{Acknowledgments}

Funding:

\hypertarget{supplemental-data}{%
\section{Supplemental Data}\label{supplemental-data}}

Supplementary Material should be uploaded separately on submission, if
there are Supplementary Figures, please include the caption in the same
file as the figure. LaTeX Supplementary Material templates can be found
in the Frontiers LaTeX folder

\hypertarget{references}{%
\section{References}\label{references}}

A reference list should be automatically created here. However it won't.
Pandoc will place the list of references at the end of the document
instead. There are no convenient solution for now to force Pandoc to do
otherwise. The easiest way to get around this problem is to edit the
LaTeX file created by Pandoc before compiling it again using the
traditional LaTeX commands.

\hypertarget{figures}{%
\section*{Figures}\label{figures}}
\addcontentsline{toc}{section}{Figures}

\begin{figure}

{\centering \includegraphics[width=160mm,height=100mm]{../data analysis/weather/Figure 1} 

}

\caption{Mean average temperature (C) and montly sum precipitation (mm) at Arlington, WI, Clay Center, NE, Grant, NE, Lincoln, NE and Macomb, IL}\label{fig:Figure-1}
\end{figure}

\begin{figure}

{\centering \includegraphics[width=150mm,height=260mm]{../data analysis/figures/combine 3} 

}

\caption{Figure caption}\label{fig:Figure-2}
\end{figure}

\hypertarget{refs}{}
\begin{CSLReferences}{1}{0}
\leavevmode\hypertarget{ref-aulakh2021}{}%
Aulakh, J. S., Chahal, P. S., Kumar, V., Price, A. J., and Guillard, K.
(2021). Multiple herbicide-resistant {Palmer} amaranth ({Amaranthus}
palmeri) in {Connecticut}: Confirmation and response to {POST}
herbicides. \emph{Weed Technology} 35, 457--463.
doi:\href{https://doi.org/10.1017/wet.2021.6}{10.1017/wet.2021.6}.

\leavevmode\hypertarget{ref-bagavathiannan2016}{}%
Bagavathiannan, M. V., and Norsworthy, J. K. (2016). Multiple-{Herbicide
Resistance Is Widespread} in {Roadside Palmer Amaranth Populations}.
\emph{PLOS ONE} 11, e0148748.
doi:\href{https://doi.org/10.1371/journal.pone.0148748}{10.1371/journal.pone.0148748}.

\leavevmode\hypertarget{ref-berger2015}{}%
Berger, S. T., Ferrell, J. A., Rowland, D. L., and Webster, T. M.
(2015). Palmer {Amaranth} ({Amaranthus} palmeri) {Competition} for
{Water} in {Cotton}. \emph{Weed Science} 63, 928--935.
doi:\href{https://doi.org/10.1614/WS-D-15-00062.1}{10.1614/WS-D-15-00062.1}.

\leavevmode\hypertarget{ref-briscoerunquist2019}{}%
Briscoe Runquist, R. D., Lake, T., Tiffin, P., and Moeller, D. A.
(2019). Species distribution models throughout the invasion history of
{Palmer} amaranth predict regions at risk of future invasion and reveal
challenges with modeling rapidly shifting geographic ranges. \emph{Sci
Rep} 9, 2426.
doi:\href{https://doi.org/10.1038/s41598-018-38054-9}{10.1038/s41598-018-38054-9}.

\leavevmode\hypertarget{ref-chahal2018}{}%
Chahal, P. S., Irmak, S., Jugulam, M., and Jhala, A. J. (2018).
Evaluating {Effect} of {Degree} of {Water Stress} on {Growth} and
{Fecundity} of {Palmer} amaranth ({Amaranthus} palmeri) {Using Soil
Moisture Sensors}. \emph{Weed Science} 66, 738--745.
doi:\href{https://doi.org/10.1017/wsc.2018.47}{10.1017/wsc.2018.47}.

\leavevmode\hypertarget{ref-farmer2017}{}%
Farmer, J. A., Webb, E. B., Pierce, R. A., and Bradley, K. W. (2017).
Evaluating the potential for weed seed dispersal based on waterfowl
consumption and seed viability. \emph{Pest Management Science} 73,
2592--2603. doi:\href{https://doi.org/10.1002/ps.4710}{10.1002/ps.4710}.

\leavevmode\hypertarget{ref-garetson2019}{}%
Garetson, R., Singh, V., Singh, S., Dotray, P., and Bagavathiannan, M.
(2019). Distribution of herbicide-resistant {Palmer} amaranth
({Amaranthus} palmeri) in row crop production systems in {Texas}.
\emph{Weed Technology} 33, 355--365.
doi:\href{https://doi.org/10.1017/wet.2019.14}{10.1017/wet.2019.14}.

\leavevmode\hypertarget{ref-guo2003}{}%
Guo, P., and Al-Khatib, K. (2003). Temperature effects on germination
and growth of redroot pigweed ({Amaranthus} retroflexus), {Palmer}
amaranth ({A}. Palmeri), and common waterhemp ({A}. rudis). \emph{Weed
Science} 51, 869--875.
doi:\href{https://doi.org/10.1614/P2002-127}{10.1614/P2002-127}.

\leavevmode\hypertarget{ref-hartzler2016}{}%
Hartzler, B., and Anderson, M. (2016). Palmer amaranth: {It}'s here, now
what? 10.

\leavevmode\hypertarget{ref-jha2010}{}%
Jha, P., Norsworthy, J. K., Riley, M. B., and Bridges, W. (2010). Annual
{Changes} in {Temperature} and {Light Requirements} for {Germination} of
{Palmer Amaranth} ({Amaranthus} palmeri) {Seeds Retrieved} from {Soil}.
\emph{Weed Science} 58, 426--432.
doi:\href{https://doi.org/10.1614/WS-D-09-00038.1}{10.1614/WS-D-09-00038.1}.

\leavevmode\hypertarget{ref-keeley1987}{}%
Keeley, P. E., Carter, C. H., and Thullen, R. J. (1987). Influence of
{Planting Date} on {Growth} of {Palmer Amaranth} ({Amaranthus} palmeri).
\emph{Weed Science} 35, 199--204.
doi:\href{https://doi.org/10.1017/S0043174500079054}{10.1017/S0043174500079054}.

\leavevmode\hypertarget{ref-kistner2018}{}%
Kistner, E. J., and Hatfield, J. L. (2018). Potential {Geographic
Distribution} of {Palmer Amaranth} under {Current} and {Future
Climates}. \emph{Agricultural \& Environmental Letters} 3, 170044.
doi:\href{https://doi.org/10.2134/ael2017.12.0044}{10.2134/ael2017.12.0044}.

\leavevmode\hypertarget{ref-klingaman1994}{}%
Klingaman, T. E., and Oliver, L. R. (1994). Palmer {Amaranth}
({Amaranthus} palmeri) {Interference} in {Soybeans} ({Glycine} max).
\emph{Weed Science} 42, 523--527.
doi:\href{https://doi.org/10.1017/S0043174500076888}{10.1017/S0043174500076888}.

\leavevmode\hypertarget{ref-kohrt2017}{}%
Kohrt, J. R., Sprague, C. L., Nadakuduti, S. S., and Douches, D. (2017).
Confirmation of a {Three}-{Way} ({Glyphosate}, {ALS}, and {Atrazine})
{Herbicide}-{Resistant Population} of {Palmer Amaranth} ({Amaranthus}
palmeri) in {Michigan}. \emph{Weed Science} 65, 327--338.
doi:\href{https://doi.org/10.1017/wsc.2017.2}{10.1017/wsc.2017.2}.

\leavevmode\hypertarget{ref-kupper2017}{}%
Küpper, A., Borgato, E. A., Patterson, E. L., Netto, A. G., Nicolai, M.,
Carvalho, S. J. P. de, Nissen, S. J., Gaines, T. A., and Christoffoleti,
P. J. (2017). Multiple {Resistance} to {Glyphosate} and {Acetolactate
Synthase Inhibitors} in {Palmer Amaranth} ({Amaranthus} palmeri)
{Identified} in {Brazil}. \emph{Weed Science} 65, 317--326.
doi:\href{https://doi.org/10.1017/wsc.2017.1}{10.1017/wsc.2017.1}.

\leavevmode\hypertarget{ref-larran2017}{}%
Larran, A. S., Palmieri, V. E., Perotti, V. E., Lieber, L., Tuesca, D.,
and Permingeat, H. R. (2017). Target-site resistance to acetolactate
synthase ({ALS})-inhibiting herbicides in {Amaranthus} palmeri from
{Argentina}. \emph{Pest Management Science} 73, 2578--2584.
doi:\href{https://doi.org/10.1002/ps.4662}{10.1002/ps.4662}.

\leavevmode\hypertarget{ref-lindsay2017}{}%
Lindsay, K., Popp, M., Norsworthy, J., Bagavathiannan, M., Powles, S.,
and Lacoste, M. (2017). {PAM}: {Decision Support} for {Long}-{Term
Palmer Amaranth} ({Amaranthus} palmeri) {Control}. \emph{Weed
Technology} 31, 915--927.
doi:\href{https://doi.org/10.1017/wet.2017.69}{10.1017/wet.2017.69}.

\leavevmode\hypertarget{ref-maclaren2020}{}%
MacLaren, C., Storkey, J., Menegat, A., Metcalfe, H., and
Dehnen-Schmutz, K. (2020). An ecological future for weed science to
sustain crop production and the environment. {A} review. \emph{Agron.
Sustain. Dev.} 40, 24.
doi:\href{https://doi.org/10.1007/s13593-020-00631-6}{10.1007/s13593-020-00631-6}.

\leavevmode\hypertarget{ref-massinga2001}{}%
Massinga, R. A., Currie, R. S., Horak, M. J., and Boyer, J. (2001).
Interference of {Palmer} amaranth in corn. \emph{Weed Science} 49,
202--208.
doi:\href{https://doi.org/10.1614/0043-1745(2001)049\%5B0202:IOPAIC\%5D2.0.CO;2}{10.1614/0043-1745(2001)049{[}0202:IOPAIC{]}2.0.CO;2}.

\leavevmode\hypertarget{ref-matzrafi2021}{}%
Matzrafi, M., Osipitan, O. A., Ohadi, S., and Mesgaran, M. B. (2021).
Under pressure: Maternal effects promote drought tolerance in progeny
seed of {Palmer} amaranth ({Amaranthus} palmeri). \emph{Weed Science}
69, 31--38.
doi:\href{https://doi.org/10.1017/wsc.2020.75}{10.1017/wsc.2020.75}.

\leavevmode\hypertarget{ref-milani2021}{}%
Milani, A., Panozzo, S., Farinati, S., Iamonico, D., Sattin, M., Loddo,
D., and Scarabel, L. (2021). Recent {Discovery} of {Amaranthus} palmeri
{S}. {Watson} in {Italy}: {Characterization} of {ALS}-{Resistant
Populations} and {Sensitivity} to {Alternative Herbicides}.
\emph{Sustainability} 13, 7003.
doi:\href{https://doi.org/10.3390/su13137003}{10.3390/su13137003}.

\leavevmode\hypertarget{ref-morgan2001}{}%
Morgan, G. D., Baumann, P. A., and Chandler, J. M. (2001). Competitive
{Impact} of {Palmer Amaranth} ({Amaranthus} palmeri) on {Cotton}
({Gossypium} hirsutum) {Development} and {Yield}. \emph{Weed Technology}
15, 408--412.
doi:\href{https://doi.org/10.1614/0890-037X(2001)015\%5B0408:CIOPAA\%5D2.0.CO;2}{10.1614/0890-037X(2001)015{[}0408:CIOPAA{]}2.0.CO;2}.

\leavevmode\hypertarget{ref-oliveira2018}{}%
Oliveira, M. C., Gaines, T. A., Patterson, E. L., Jhala, A. J., Irmak,
S., Amundsen, K., and Knezevic, S. Z. (2018). Interspecific and
intraspecific transference of metabolism-based mesotrione resistance in
dioecious weedy {Amaranthus}. \emph{The Plant Journal} 96, 1051--1063.
doi:\href{https://doi.org/10.1111/tpj.14089}{10.1111/tpj.14089}.

\leavevmode\hypertarget{ref-oliveira2021a}{}%
Oliveira, M. C., Giacomini, D. A., Arsenijevic, N., Vieira, G., Tranel,
P. J., and Werle, R. (2021). Distribution and validation of genotypic
and phenotypic glyphosate and {PPO}-inhibitor resistance in {Palmer}
amaranth ({Amaranthus} palmeri) from southwestern {Nebraska}. \emph{Weed
Technology} 35, 65--76.
doi:\href{https://doi.org/10.1017/wet.2020.74}{10.1017/wet.2020.74}.

\leavevmode\hypertarget{ref-page2021}{}%
Page, E. R., Nurse, R. E., Meloche, S., Bosveld, K., Grainger, C.,
Obeid, K., Filotas, M., Simard, M.-J., and Laforest, M. (2021). Import
of {Palmer} amaranth ({Amaranthus} palmeri {S}. {Wats}.) Seed with sweet
potato ({Ipomea} batatas ({L}.) {Lam}) slips. \emph{Can. J. Plant Sci.},
CJPS-2020-0321.
doi:\href{https://doi.org/10.1139/CJPS-2020-0321}{10.1139/CJPS-2020-0321}.

\leavevmode\hypertarget{ref-price2011}{}%
Price, A. J., Balkcom, K. S., Culpepper, S. A., Kelton, J. A., Nichols,
R. L., and Schomberg, H. (2011). Glyphosate-resistant {Palmer} amaranth:
{A} threat to conservation tillage. \emph{Journal of Soil and Water
Conservation} 66, 265--275.
doi:\href{https://doi.org/10.2489/jswc.66.4.265}{10.2489/jswc.66.4.265}.

\leavevmode\hypertarget{ref-ritz2015}{}%
Ritz, C., Baty, F., Streibig, J. C., and Gerhard, D. (2015).
Dose-{Response Analysis Using R}. \emph{PLOS ONE} 10, e0146021.
doi:\href{https://doi.org/10.1371/journal.pone.0146021}{10.1371/journal.pone.0146021}.

\leavevmode\hypertarget{ref-sauer1957}{}%
Sauer, J. (1957). Recent {Migration} and {Evolution} of the {Dioecious
Amaranths}. \emph{Evolution} 11, 11--31.
doi:\href{https://doi.org/10.2307/2405808}{10.2307/2405808}.

\leavevmode\hypertarget{ref-sauer1972}{}%
Sauer, J. D. (1972). The dioecious amaranths: A new species name and
major range extensions. \emph{Madroño} 21, 426--434. Available at:
\url{http://www.jstor.org/stable/41423815}.

\leavevmode\hypertarget{ref-scott2011}{}%
Scott, D. (2011). The {Technological Fix Criticisms} and the
{Agricultural Biotechnology Debate}. \emph{J Agric Environ Ethics} 24,
207--226.
doi:\href{https://doi.org/10.1007/s10806-010-9253-7}{10.1007/s10806-010-9253-7}.

\leavevmode\hypertarget{ref-ward2013}{}%
Ward, S. M., Webster, T. M., and Steckel, L. E. (2013). Palmer
{Amaranth} ({Amaranthus} palmeri): {A Review}. \emph{Weed Technology}
27, 12--27.
doi:\href{https://doi.org/10.1614/WT-D-12-00113.1}{10.1614/WT-D-12-00113.1}.

\leavevmode\hypertarget{ref-yu2021}{}%
Yu, E., Blair, S., Hardel, M., Chandler, M., Thiede, D., Cortilet, A.,
Gunsolus, J., and Becker, R. (2021). Timeline of {Palmer} amaranth
({Amaranthus} palmeri) invasion and eradication in {Minnesota}.
\emph{Weed Technology}, 1--31.
doi:\href{https://doi.org/10.1017/wet.2021.32}{10.1017/wet.2021.32}.

\end{CSLReferences}

\end{document}
